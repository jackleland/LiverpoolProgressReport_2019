%----------------------------------------------------------------------------------------
%	PACKAGES AND OTHER DOCUMENT CONFIGURATIONS
%----------------------------------------------------------------------------------------

\documentclass[a4paper, 12pt]{article} % Font size (can be 10pt, 11pt or 12pt) and paper size (remove a4paper for US letter paper)

\usepackage[protrusion=true,expansion=true]{microtype} % Better typography
\usepackage{graphicx} % Required for including pictures
\usepackage{wrapfig} % Allows in-line images
\usepackage{multirow}

%\usepackage{mathpazo} % Use the Palatino font
\usepackage[T1]{fontenc} % Required for accented characters
\linespread{1.04} % Change line spacing here, Palatino benefits from a slight increase by default


% Included by Jack
\usepackage[
%width=16cm, 
left=1.48cm,
right=1.48cm,
top=2.45cm, 
bottom=2.45cm
]{geometry}
\usepackage{hyperref}
\usepackage{float}
\usepackage{fancyhdr}
\usepackage{cleveref}
\usepackage{subfig}
\usepackage[super]{nth}
\usepackage{setspace}

% for drawing lines
\usepackage{tikz}

% for gantt chart
\usepackage{pgfgantt}
\usepackage{xcolor}
\usepackage{color}
\definecolor{lblu}{RGB}{0,112,210}
\definecolor{blu}{RGB}{0,56,158}
\definecolor{gren}{RGB}{132,235,176}
\definecolor{purp}{RGB}{213,174,212}
\definecolor{yell}{RGB}{255,255,0}

\usepackage[
backend=bibtex,
style=numeric,
citestyle=numeric,
sorting=none,
maxnames=2
]{biblatex}
\addbibresource{mendeley.bib}
% Some field suppression via options
\ExecuteBibliographyOptions{isbn=false,url=false,doi=false,eprint=false}

\graphicspath{{images/}}

% Header and footer commands
\pagestyle{fancy}
\fancyhf{}
\rhead{Jack Leland}
\lhead{\textsc{Third Year Progress Report}}
\cfoot{\thepage}

\makeatletter
\renewcommand{\@listI}{\itemsep=0pt} % Reduce the space between items in the itemize and enumerate environments and the bibliography
\renewcommand*{\bibfont}{\tiny}


\renewcommand{\maketitle}{ % Customize the title - do not edit title and author name here, see the TITLE block below
\begin{minipage}{0.3\linewidth}
	\begin{flushleft}
%		\vspace{-95pt}
		\vspace{-12ex}
		\hspace{-17pt}\includegraphics[width=1\linewidth]{Logos/Liverpool.jpg} \\
		\vspace{10pt}
		\hspace{-20pt}\includegraphics[width=1\linewidth]{Logos/cdt_logo_text_black.png} \\
		
		
%		\vspace{10pt}
%		\hspace{-20pt}\includegraphics[width=1\linewidth]{CCFE.jpg} 		
		\vfill
	\end{flushleft}
\end{minipage}
\hfill
\begin{minipage}{0.65\linewidth}
	\begin{flushright} % Right align
		\vspace{-5ex}
		{\LARGE\@title} % Increase the font size of the title
		
		\vspace{10pt} % Some vertical space between the title and author name
		
		{\large\@author} % Author name
		
		\vspace{20pt}
		\@date % Date
		
%		\vspace{40pt} % Some vertical space between the author block and abstract
	\end{flushright}
\end{minipage}
\vspace{15pt} % Some vertical space between the abstract and first section
}
\newcommand{\spacer}{\rule{0pt}{3ex}}

%----------------------------------------------------------------------------------------
%	TITLE
%----------------------------------------------------------------------------------------

\title{\textbf{Third Year Progress Report} \\
		Measurements and particle-in-cell simulations of MAST-U flush-mounted Langmuir probes} % Subtitle

\author{\textsc{Jack Leland} % Author
	\\ {\textit{University of Liverpool, Brownlow Hill, Liverpool, L69 3GJ}}
	\\ \href{mailto:j.leland@liverpool.ac.uk}{j.leland@liverpool.ac.uk}
	} % Institution

\date{\today} % Date

%----------------------------------------------------------------------------------------

\begin{document}

\maketitle % Print the title section
\thispagestyle{plain}

%----------------------------------------------------------------------------------------
%	ABSTRACT AND KEYWORDS
%----------------------------------------------------------------------------------------

%\renewcommand{\abstractname}{Summary} % Uncomment to change the name of the abstract to something else

\begin{abstract}
Novel divertor configurations are being explored as a possible solution to the heat exhaust problem in tokamaks.
Detailed knowledge of temperature and density in these divertors is therefore essential when testing the effectiveness of the heat mitigation provided by the configurations.
850 Langmuir probes have been installed in the plasma facing components (PFCs) of MAST-U however their analysis will be complex owing to the challenges associated with the sensitivity of flush mounted probes to field line incidence angles. 
Detailed 2 and 3 dimensional particle-in-cell simulations are being carried out, using the particle-in-cell code SPICE, of a variety of different Langmuir probe tip geometries.
By simulating the voltage sweep of a probe in a prescribed plasma environment, the measured current-voltage characteristic can be compared to the specified temperature and density. 
Two experimental trips to Magnum-PSI in the Netherlands were also carried out to verify that the angled tip design, used on the probes in MAST-U, works at the grazing angles of incidence expected in the MAST-U divertor. 
The data from these experiments and the simulations will be compared and used to make modifications to the current Langmuir probe theory to properly account for field line angle effects and 3D tip geometries.
The updated model can then be applied to the interpretation of flush mounted Langmuir probe data from MAST-U.
 
\end{abstract}

%\hspace*{3,6mm}\textit{Keyswords:} lorem , ipsum , dolor , sit amet , lectus % Keywords

%\vspace{15pt} % Some vertical space between the abstract and first section

%\tableofcontents
%------------------------------------------------------------------------------------
%	ESSAY BODY
%------------------------------------------------------------------------------------


\section{\label{sec:intro}Introduction}

\subsection{\label{subsec:tokamaks}Tokamaks}
	Fusion is a potential candidate for base-load commercial energy production due to its lack of greenhouse gas emissions, high power output, abundant fuel and lack of long-lived radioactive waste.
%	A Deuterium-Tritium (DT) fuel mixture must be heated to high enough temperatures to create collisions of sufficient energy to allow fusion to take place. 
%	The equation for this reaction is:
%	\begin{equation}
%	\label{eq:DTEq}
%	{}^{2}_{1}D + {}^{3}_{1}T \rightarrow   {}^{4}_{2}He (3.5 MeV) + {}^{1}_{0}n (14.1 MeV)
%	\end{equation} 
%	The temperatures required ($\sim$ 200 eV for DT) will ionise the fuel mixture into a plasma, meaning that the individual charged components can be manipulated and confined by magnetic fields.
	One of the methods proposed for confining the fusion reaction is the use of magnetic fields, known as magnetic confinement fusion (MCF).
	This can be done using several techniques, the most widely used of which is the Tokamak; a toroidal magnetic device comprised of a combination of toroidal and poloidal magnetic fields in a helical shape. 
	This setup is being developed with the aim of confining the plasma for a long enough time to allow net energy to be extracted from the fusion reaction and delivered to the grid.  
	
\subsection{\label{subsec:exhaust}The Exhaust Problem and MAST-Upgrade}
	A necessary consequence of the magnetic containment of a plasma within a vessel, such as a tokamak, is the existence of open field lines, i.e. field lines which impinge on surfaces. 
	Particles orbiting along open field lines will therefore impact on these surfaces, meaning these field lines create an avenue for heat \& particles to flow out of the plasma and be deposited onto surfaces.
	This can be helpful as it allows the removal of impurities - notably helium `ash', a by product of the Deuterium-tritium (DT) fusion reaction - however it is usually necessary to physically separate the impinged surfaces from the core plasma to reduce the opposing flow of sputtered impurities into the core which radiate away heat. 
	This can be done with either a limiter or a divertor\cite{Wesson2011}, with modern devices almost exclusively using divertors because of (a) the reduced sputtering of impurities into the core, and (b) access to H-mode - a mode of operation with higher confinement of particles in the core \cite{Itoh1989}.
	
%	  usually done with use of a divertor (\cref{fig:divertor}), where the targets are separated from the core plasma by a magnetic null point, or X-point.
%	The points at which the separatrix - the line separating the closed and open field lines - hit the target plates are called strike points.
%	These strike points will experience the highest heat and particle fluxes from the core plasma as particles cross the separatrix and flow towards the target through the scrape-off layer.
%	
%	Divertors have been used for decades with great success, allowing for more stable plasmas to be created. 
	However, when scaling Tokamaks to the size required for commercial electricity generation, the divertor target plates will experience very high heat fluxes because of the high core temperature required.  
	Dealing with and mitigating these heat fluxes is one of the key challenges currently standing in the way of commercial fusion power for toroidal devices\cite{Romanelli2012}. 	 
	
%\subsection{\label{subsec:mast}MAST-Upgrade}
	\begin{wrapfigure}[17]{rt}{6cm}
		\centering
		\vspace{-10pt}
		\includegraphics[width=1.0\linewidth]{IllustrativeFigures/divertor.jpg}
		\caption{\label{fig:divertor}A diagram of the regions within a tokamak plasma, including the separatrix and the divertor, separated from the core plasma by the X-point.}
	\end{wrapfigure}
	A proposed mitigation technique for the heat exhaust problem is the use of divertor configurations which allow for reduced heat flux to plasma facing components (PFCs). 
	MAST-Upgrade (MAST-U) at CCFE has been designed as an experiment to test several aspects of divertor physics, namely the use of a closed divertor with Super-X capability \cite{Valanju2009}.
	The closed aspect of the divertor means that it is separated from the main chamber by a baffle, and therefore more effective at restricting the flow of impurities and neutrals into the core, improving confinement. 
	The higher neutral pressure in the closed divertor also enhances the amount of energy loss via radiation, dissipating the power hitting the target plates.
	The Super-X configuration reduces heat flux further by extending the plasma out radially, increasing both the size of the strike point, so the same power is spread over a wider area; and the flight time of particles in the divertor, allowing them to lose energy over a longer period of time.
%	Lower neutral pressure in the core results in higher confinement and easier access to H-mode.
%	A detached plasma should also be easier to access, which involves having an increased neutral density in the divertor which increases radiation of energy from the plasma before it hits the target.
%	\begin{wrapfigure}[16]{rt}{7cm}
%		\centering
%		\vspace{-1.65cm}
%		\subfloat[]{\includegraphics[width=0.4\linewidth]{IllustrativeFigures/MastDivertorZoom.png}}
%		\hspace{10pt}
%		\subfloat[]{\includegraphics[width=0.4\linewidth]{IllustrativeFigures/LPLocations.png}}
%		\caption{Diagrams of (a) the magnetic flux of the Super-X configuration in MAST-U and (b) the locations of the Langmuir probe arrays in the MAST-U divertor (right).}
%		\label{fig:mast_divertor}
%	\end{wrapfigure} 

%	The combination of the closed divertor and the Super-X should allow for more effective control of the heat flux to the divertor tiles.
%	It is therefore necessary to understand, with great detail, what goes on in the MAST-U divertor when it is tested experimentally to ensure that the models predicting the behaviour of the plasma are valid. 
%	We therefore require detailed measurements of the density and temperature of the plasma in this region, which must be gained through diagnostic techniques.

\subsection{\label{subsec:lps}Langmuir Probes and Flush Mounting}
	Langmuir probes are a widely used diagnostic in plasma physics to measure electron temperature ($T_e$) and density ($n_e$). 
	In tokamaks they are limited to measuring plasma parameters near to the PFCs and the edge due to the extreme nature of the plasmas involved. 
	They are thus regularly flush mounted to reduce the incident heat flux to the probe tips and reduce erosion. 
	Flush mounted probes (FMPs) are however notoriously difficult to interpret in strongly magnetised plasmas \cite{Matthews1994}, notably because of the strong dependence on incidence angle of the magnetic field on the resultant shape of the current-voltage (IV) characteristic.
	
	At grazing angles of incidence, routinely found in tokamaks, sheath expansion becomes a dominant factor in the determination of the effective collection area ($A_{eff}$) of the probe \cite{Bergmann1994}. 
	Small uncertainties in the angle can therefore have large effects on the amount of sheath expansion experienced by the probe and therefore the shape of the IV characteristic.
	On top of this, magnetised probe theory is only valid when the projected probe extent is greater than the local Debye length and Larmor radius \cite{Gunn1995}. 
	The form of the IV characteristic, taking these effects into account, was developed by Bergmann \cite{Bergmann2002} and is as follows:
	\begin{equation}
	\label{eq:ivcurvesheathexp}
	\frac{I}{I_{i,sat}} =  1 + a |V|^{\frac{3}{4}} - \exp\left({-V}\right) \qquad \textnormal{with} \qquad  V = \bigg(\frac{e(V_p - V_f)}{k_B T_e}\bigg)
	\end{equation}
	where $I$ is the measured current, $I_{i,sat}$ is the ion saturation current, $V_p$ and $V_f$ are the probe and floating potentials respectively, $e$ is the charge on an electron and $k_B$ is the Boltzmann constant. 
	Bergmann's, and later Murphy-Sugrue's \cite{Murphy-Sugrue2017}, contribution to this equation is the addition of a sheath expansion term, $a$, which takes into account the incidence angle of the magnetic field $\theta_{\perp}$. 
	It is of the form
	\begin{equation}
	\label{eq:sheathexpparam}
	a =  \frac{c_1 + c_2  \cot{(\theta_{\perp})}}{\sin^{\frac{1}{2}}{(\theta_{\perp}})} \frac{\lambda_D}{L + g}
	\end{equation}
	where $c_1$ and $c_2$ are coefficients of the lateral and frontal expansion of the sheath respectively, $\lambda_D$ is the Debye length of the plasma, $L$ is the length of the probe and $g$ is the size of the gaps either side of the probe.
	
	These issues have been previously mitigated on JET\cite{Monk1996} and DIII-D by angling the probe tip with respect to the incident magnetic field to increase the projected probe area and therefore lower the dependence on small changes of angle. 
	A similar probe tip has been designed for use in the divertor of the MAST-U tokamak at CCFE \cite{Harrison}, with 850 probes of this type installed throughout the PFCs.
	
\subsection{\label{subsec:motivation}PhD Project}
	As the collection area of the probe is essential to the calculation of the current density of the flux tube, getting measurements of $n_e$ becomes highly dependent on knowing the incidence angle and using this to calculate the area of the probe.
	This is also true for $T_e$, but to a lesser extent, as this is more dependent on orbits than small changes in collection area. 
	The main aim of this PhD project is to simulate the FMPs that will be used in the PFCs of MAST-U, using a 2D/3D3V particle-in-cell (PIC) simulation code called SPICE \cite{Komm2011, Komm2013}. 
	These simulations are currently being run on the CUMULUS supercomputing cluster located at CCFE, and the MARCONI supercomputing cluster hosted by Eurofusion\cite{Voitsekhovitch2018} which boasts  improved scalability over CUMULUS.
	These simulations are done with the aim of better understanding the IV curves produced in a wide range of conditions to therefore allow improvements to be made to the analysis model for MAST-U experimental data.
	Simulation results will also facilitate improvements to be made to the current model of FMP analysis by extending the models into 3-dimensions - as advancements in supercomputing now allow for efficient computation of the complex physics happening around a probe tip.
	This will allow for a more rigorous treatment of the probe tip geometry to be formulated than that found in \cref{eq:sheathexpparam} and therefore aid in generalising the analysis of FMPs.

	
%------------------------------------------------------------------------------------
%	Magnum Measurements
%------------------------------------------------------------------------------------

\section{\label{sec:mag}Further Experiments at Magnum-PSI}
\subsection{\label{ssec:mag-prev}Previous Experiment}
    In my second year I undertook an experiment on a linear device, aiming to investigate the effects of neutral gas pressure, and thereby detachment, on the IV characteristics of MAST-U probes before they were used in the upcoming campaign.
    Measurements were taken on the linear device Magnum-PSI at DIFFER in Eindhoven, NL, using a 4-probe array designed for the Divertor Science Facility (DSF) at MAST-U \cite{Elmore2012} with a range of control parameters including incidence angle of magnetic field. 
	A secondary goal of this experiment was to test that these probes performed as expected with regards to minimising sheath expansion and producing expected IV characteristics.
    This experiment was left incomplete, so a return experiment was carried out in May and June of this year with a more extensive set of measurements planned. 
    
    The previous experiments had three major findings: the DSF probe head had a problematic choice of material for the pin holder encasing the graphite probes, a detachment front could arguably be seen moving along the plasma beam with increasing neutral gas pressure, and the plasma pressure seemed to be in agreement with the measurements from the Thomson scattering (TS) system on Magnum-PSI - though it is not clear that this wasn't just due to fortuitous cancellation of the overestimated of temperatures (by a factor of 2-4) and underestimated densities (by a factor of 0.25-0.5) compared to the TS measurements.
    
    Extensive analysis was carried out on this data to try to explain and mitigate the overestimation of temperature and underestimation of density compared to the TS measurements. 
    Multiple avenues were explored, including an alternative 4-parameter model \cite{Gunn1995}, a hot electron tail disproportionately affecting measurement of the bulk temperature \cite{Stangeby1995a}, multiple species interfering with the current collection, supersonic drifts within the beam, enhanced filtering of oscillations within the plasma, accounting for non-linear sheath dynamics if fast oscillations were present; but none of these were able to explain or counteract the overestimation of temperature. 
    Three main explanations remained following this process: (a) a non-maxwellian electron energy distribution function (EEDF), (b) incomplete characterisation and calibration of the electronics used, (c) temperatures too low for the Langmuir probes to work properly in a machine environment \cite{Stangeby2000}.
    Directly measuring the EEDF with a Langmuir probe to confirm (a) is difficult in a magnetised plasma, principally because the second derivative is required and this is not easy to obtain due to experimental noise. 
    It is possible to measure the EEDF with the TS system, and this is being looked into by Magnum staff. 
    (b) and (c) were specified by the experimental design however, so a return experiment was planned to mitigate these two potential problems. 
    This would be done through relaxation of the limits specified by needing MAST-U equivalent magnetic field strength (0.5-0.8T) and plasma species, allowing higher temperatures ($ T_e >=$ 5eV).
    Point (c) could be addressed through the use of alternative electronics, for example the MAST-U probe electronics.
	
	\begin{wrapfigure}[15]{rt}{10cm}
		\vspace{-10pt}
		\centering
		\begin{tikzpicture}
			\node (zoomin){ \includegraphics[width=2.1cm]{MAST-U/dsfProbesPlanCroppedLabelled.png}};	
			\node [right=of zoomin] (zoomout){\includegraphics[width=0.4\linewidth]{Magnum/ProbeMountFlip.png}};
%			\draw (2.25,-2.15) to (5.0,-0.35);
%			\draw (2.25,2.15) to (5.0,0.15);
%			\draw (5.0,0.15) rectangle (5.5,-0.35);
			\draw (1.05,-2.34) to (4.2,-0.26);
			\draw (1.05,2.34) to (3.78,0.0);
%			\draw (4.98,-0.0) -- (5.4,-0.26) -- (5.65,-0.21) -- (5.23,0.05) -- cycle;
			\draw (3.78,-0.0) -- (4.2,-0.26) -- (4.45,-0.21) -- (4.03,0.05) -- cycle;
		\end{tikzpicture}
		\caption{\label{fig:dsfprobe}\textit{(left)} CAD schematic of DSF probe head, with four probe tips of differing area and shape. \textit{(right)} CAD schematic of designed mounting system for installing the DSF probe head onto Magnum's flat, rotatable target holder.}
		\vspace{-10pt}
	\end{wrapfigure}
	
	The remainder of this section describes the second experimental setup, analysis techniques used and subsequent results, ending with a discussion on the consequences of this for the interpretation of Langmuir probe data in MAST-U. 
	Note that this analysis is still ongoing. 

\subsection{\label{ssec:mag-exp-method}Experimental Method}
	The probe head used in this experiment was a 4-probe array designed for use in the DSF at MAST-U: a sample testing facility installed within the MAST-U divertor which facilitates the exchange of samples or electrical probes while maintaining vessel vacuum.
	 
	This probe head has 4 Langmuir probes with tips of different geometry: a standard MAST-U tip, a tip with twice the geometric area of a MAST-U tip, a tip with half the area of a MAST-U tip, and a flush cylindrical probe with similar area to the standard probe. 
	(For reference these will be labelled L, B, S, and R respectively, taken from their designation in the Magnum beam-hall). 
	The first 3 probes are right-trapezoidal in geometry (\cref{fig:dsfprobe}) with the surface of the trapezium angled 10$^{\circ}$ relative to the divertor tile it is recessed into.

	The four probes are electrically isolated from each other by a ceramic pin holder - changed following the previous experiments findings that PEEK was not a suitable material - and the leading edges of the probes shadowed from plasma exposure by a graphite shell around the whole assembly. 
	The distances between the probes and the graphite shell were chosen during a design study \cite{Harrison}, such that the probe's leading edges are shadowed at incidence angles up to 10$^{\circ}$ - the average expected operating regime of MAST-U in Super-X configuration.	
	The same specialised mounting assembly which was designed and manufactured for the previous experiment was used again to hold the DSF probe head in place against the Magnum flat target holder (see \cref{fig:dsfprobe}).
	This target holder is water cooled and could tilt to give a range of magnetic incidence angles onto the probes.
	The probes were prepared by vacuum bake at 200$^{\circ}$C for 5 hours, to minimise outgassing during operation.
	
	The intended electronics, as discussed in \cref{ssec:mag-prev}, were to be the power supply, amplifier and FPGA multiplexing unit to be used for the probes on MAST-U \cite{Lovell2017}. 
	These would have allowed simultaneous measurements to be taken on all 4 probes and faster sweeping - meaning that better time resolution for plasma oscillations would be available. 
	Unfortunately this was not possible due to an unknown fault that occurred in the multiplexing unit while attempting to use the electronics at Magnum-PSI. 
	As such the electronics used previously - an isolational amplifier with a KEPCO power supply and shunt resistors for current measurement - had to be used again. 
	Special care was taken to more thoroughly characterise these electronics however, with proper accounting carried out for internal resistances and an improved measure of the voltage/current measured implemented in analysis.
	
%	The voltage sweep on the probes was produced by an Agilent 3312A arbitrary function generator, the waveform used was a 20Hz triangle wave varying between -100 and +10 V. 
%	This was amplified by a KEPCO 100-4M 100V bipolar operational power supply and applied to two of the four probes at a time. 
%	The current from the probes was measured by using a dual-channel isolational amplifier to observe the voltage drop across two 1$\Omega$ resistors. 
%	All of these signals were then digitised by a National Instruments NI PXI-5105 digitiser, with the voltage signal attenuated 10x to be sampled by the digitiser's 10V range.
%	The digitiser sampled at a minimum rate of 100kHz.
	
	 
	The probes were exposed to hydrogen, helium and deuterium plasmas with a range of temperatures and densities, attempting to replicate the conditions within the MAST-U divertor while also maximising temperature to, as far as possible, access the plasma regime ($>5$eV) best suited to probe measurement in large devices.
% 	Due to the limitations of the source \cite{DeTemmerman2015a}, the temperatures could not be pushed higher than 3eV while maintaining low enough heat fluxes to prevent secondary electron emission from the LPs\cite{Stangeby2000}. 
	The angle was varied in most cases between 0 and 10$^{\circ}$, which is the range of angles which are predicted be problematic for probe interpretation in MAST-U.
	The experimental time was primarily used to take these incidence angle scans at a variety of different plasma parameters, but other independent parameter scans were carried out:
	\begin{itemize}
	    \item Densities in a range from 1 $\times 10^{19}$ to 1 $\times 10^{21}$ m$^{-3}$. 
	    \item Temperatures in a range from 1.5eV to 9.5eV.
	    \item Magnetic field strength from 0.5T to 1.5T
	\end{itemize}
	Note that the magnetic field strength on MAST-U will be limited to 0.8T, but the previous experiments showed that this did not allow plasmas with $T_e$ above 5eV to be created on Magnum, so this requirement was relaxed.
	A neutral density scan was also performed, as a benchmark to previous measurements, by increasing pressure in the target chamber with hydrogen gas puffing at the target location.
	As per the previous experiment, the plasma was also diagnosed with the well characterised \cite{} Thomson scattering (TS) system in almost all shots.
	The infra-red camera and optical emission spectroscopy measurements were unable to be used in this set of experiments because of the alternative probe orientation chosen - the array was positioned parallel to the TS laser. 
	This unfortunately meant that the Magnum target holder and mount were blocking the line of sight for both of these experiments, so further comparison than the TS measurements is unavailable for the majority of measurements.
	
	A fortunate advantage over the previous experiment was the improved beam dump installed in Magnum-PSI last year, which now allows for the probes to be in position when the beam dump is pulled out of the plasma beam. 
	This therefore allows the probes to be exposed to the plasma for the minimum amount of time to get a TS measurement, compared to the previous experiment where additional time, on the order of 10 seconds, was required to get the probes into position.
% 	\begin{wrapfigure}[23]{lt}{8.5cm}
% 		\center{
% 			\vspace{-25pt}
% 			\includegraphics[width=0.90\linewidth]{PRYear2/probeMountGeometry2.png}	
% 			%			\hspace{15pt}
% 		}
% 		\vspace{-5pt}
% 		\caption{\label{fig:schematic}Schematic diagram showing the location of the DSF probes and mounting system in the Magnum-PSI beam, as well as the relative difference between the IR and TS probe measurement positions, with distances in mm.}
% 		\vspace{-15pt}
% 	\end{wrapfigure}
% 	The TS measured plasma 30mm closer to the source than the probes, and the IR camera was focussed 20mm in the same direction: these measurements were therefore mutually exclusive (see \cref{fig:schematic}) so each set of plasma parameters had to be measured twice at two different axial positions.
% 	The probes were pushed into position by Magnum's TEAC delivery system for 5-15 seconds at a time with 5 seconds of traversal time either side of this stationary window. 
	
% \subsection{DSF Redesign}
% %
% %	\begin{figure}[b!]
% %		\center{
% %			\vspace{-15pt}
% %			\includegraphics[width=1.0\linewidth]{CollabReport/threeStageFit_cropped.pdf}
% %		}
% %		\vspace{-15pt}
% %		\caption{The 3 stages of fitting used in analysis of IV characteristics: the straight saturation region fit (top), the fixed $I_{sat}$ fit (middle) and the full, freely varying fit (bottom).}
% %		\label{fig:threeStageFit}
% %		\vspace{-15pt}
% %	\end{figure}
% 	A key finding from the experiment was the discovery of two design flaws with the DSF probe-head stemming from the choice of PEEK as the pin-holder material.
% 	Firstly, upon vacuum-baking it was evident that there was insufficient space between the graphite shell and the PEEK pin-holder to account for the difference in thermal expansion coefficient of the two materials.
% 	This led to the cracking of the graphite shell into 3 pieces.
% 	The design for the DSF has been updated, with the PEEK pin-holder being replaced with ceramic to prevent this from occurring.
% 	Secondly, the surface temperatures of the probes and graphite shell were high enough to cause the PEEK to melt.
% 	Molten PEEK then seeped into the gaps between the probes and the graphite shell, becoming exposed to plasma and burning. 
% 	This PEEK ash was then sufficiently electrically conductive that it created a short between two of the probes and the graphite shell, prohibiting their use until they could be taken out of the machine and the conductive material removed.
% 	The surface temperature of the assembly was much higher than it will be in MAST-U due to both the longer exposure times of the Magnum setup and the unfortunate existence of an exposed leading edge on the interface between the mounting device and the DSF probe head. 
% 	The latter was confirmed by the appearance of a hotspot on the IR camera of up to 1600$^{\circ}$C depending on incidence angle.
% 	Both of these will not be present when operating in MAST-U however, as the heat fluxes will be much lower, but the change to a ceramic pin-holder has also fixed this problem for future experiments on Magnum-PSI. 
% 	The leading edge on the probe mount will require modifications to the Tungsten back-plate to remedy in future experiments.

\subsection{Results}
% 	\begin{figure}[b!]
% 		\center{
% 			\vspace{-20pt}
% 			\subfloat[]{\label{fig:threeStageFit}\includegraphics[width=0.485\linewidth]{CollabReport/threeStageFit_cropped.pdf}}
% 			%			\hspace{2pt}
% 			\subfloat[]{\vspace{-1pt}\label{fig:ivComparison}\includegraphics[width=0.513\linewidth]{PRYear2/ivComparison2.png}}
% 		}
% 		\vspace{-0pt}
% 		\caption{
% 			(a) The 3 stages of fitting used in analysis of IV characteristics: the straight saturation region fit (top), the fixed $I_{sat}$ fit (middle) and the full, freely varying fit (bottom).
% 			(b) Comparison of fitted IV curve to IV curves generated using the 	analytical formula from measured TS values. 
% 			The 'Analytical from TS' shows a much sharper knee associated with lower $T_e$, but approximately equivalent $I_{sat}$. 
% 			%			Most interestingly, the 'Analytical - measured' curve in red shows that the sheath expansion parameter calculated using coefficients $c_1$ and $c_2$ from \cite{Murphy-Sugrue2017a} slightly underestimates the amount of sheath expansion observed from the DSF probe measurements. 
% 			%			This can be seen in the difference in gradient in the ion saturation region of the IV curve.
% 		}
% 		\vspace{-15pt}
% 	\end{figure}
% 	Due to a fault with the Langmuir probe electronics, the most reliable data was obtained during the scan of neutral gas puffing - as such these results will be discussed in detail.
% 	These measurements were taken at an incidence angle of 10$^{\circ}$.
% 	The half area probe (S) is used in all measurements discussed as it was approximately in the centre of the plasma beam and did not short during the experiment.
% 	A low-pass Butterworth filter with a critical frequency of 4kHz was applied to the voltage signal to remove high frequency noise.
    The IV characteristics themselves were sweep-averaged over the course of a `shot', which was taken to be the point between the raising and lowering of the beam dump where the probes were exposed to steady state plasma. 
    Arcing was rare, but any instances of sharp increase in current or voltage were assumed to be an arc and filtered out. 
	
	IV characteristics were analysed using the 3-stage method developed on MAST, with a full 4-parameter model used throughout (\cref{eq:ivcurvesheathexp}) - parameters being $T_e$, $I_{sat}$, $V_f$ and $a$.
	The ion saturation region of the IV curve is fit first to get an initial value for ion saturation current.
	This initial value of $I_{sat}$ is then kept fixed in a second, 4-parameter fit to the whole IV curve. 
	Finally, a full, freely varying 4-parameter fit is carried out, with initial values taken from the second fit, to get final parameter values and uncertainties.
% 	Comparisons were made between a 3- and 4-parameter fit and only small differences in the fitted value of $I_{sat}$ were found, which provides good evidence that the DSF probe tip's tilted design is successfully reducing sheath expansion effects at larger angles.
	Densities were calculated from the fitted parameters using the definition of ion saturation current $I_{sat} = n_e e c_s A_{eff}$, 
	%	\begin{equation}
	%	% Modified I_sat definition
	%	\label{eq:isateffective}
	%	I_{sat} = n_e e c_s A_{eff}
	%	\end{equation}
	where $c_s$ is the sound speed and $A_{eff}$ is the effective collection area of the probe - in this case taken as the projected area along the field line, found by extending the derivation of the exposed probe extent (in \cite{Harrison}) to right trapezoidal geometry.
	The form of $A_{eff}$ for probes S, L, and B, is then given by:
	\begin{equation}
	    % IV Curve
    	\label{eq:a_eff_alternate}
    	A_{eff} = \frac{1}{2} \bigg(a + b - \frac{d}{\tan{\theta_{f}}} \bigg)\bigg(\frac{L}{\cos{\theta_{p}}} - d\bigg) \sin{\Big[|\theta_{\perp}| + |\theta_{p}|\Big]} + b \cos{\theta_{p}}\Big(d_{\phi}\tan{\theta_{\perp}} - d_{\perp}\Big)
	\end{equation}
	\begin{equation}
    	% IV Curve
    	\label{eq:d}
    	d = \frac{d_{\perp} - d_{\phi}\tan{\theta_{\perp}}}{ \sin{\theta_{p}} + \tan{\theta_{\perp}}\cos{\theta_{p}}}
	\end{equation}
	where $\theta_{\perp}$ is the incidence angle of the magnetic field relative to the tile plane, $\theta_{p}$ is the angle of the probe tip relative to the tile plane and $\theta_{f}$ is the 'flow angle' which is the angle of the right-trapezoid of the probe tip.
	For probe R - the flush cylindrical probe - this is given by:
	\begin{equation}
        \label{eq:cylinder_A_coll}
        A_{eff} = \bigg(\pi - \theta_{c} - \frac{1}{2}\sin{2\theta_{c}}\bigg) r^2\sin{\theta_{\perp}}
    \end{equation}
	where $\theta_{c}$ is half the angle of the shadowed chord of the circular probe cross-section.
	$c_s$ is calculated from $c_s = \sqrt{\frac{e(T_e + T_i)}{m_i}}$ and $T_e = T_i$ assumed. 
	
	The data set gathered is large, so the initial focus of analysis was on ensuring the calibration of the probes was done correctly using the improved characterisation and calibration shots on resistors of known resistance. 
	The first set of shots to be analysed was the axial scan - a series of measurements taken with the S and L probes at different positions along the beam relative to the fixed position of the TS laser. 
	These were all taken at an incidence angle of 10$^{\circ}$, with source input parameters (source current and gas flow) kept constant throughout. 
	As the axial position scan necessitated moving the probe head, mount and TEAC arm through the TS laser position, only a single reference TS measurement was taken to which all probe measurements were compared (see \cref{fig:mag-axial-scan})
	\begin{figure}[!tb]
		\center{
% 			\vspace{-20pt}
			\includegraphics[width=0.8\linewidth]{axial_scan_2.png}
		}
		\vspace{-20pt}
		\caption{
			Graph showing the $T_e$ (top) and $n_e$ (bottom) measured by the S and L DSF probes (right), alongside the reference TS profile measurement, taken at a location of 200mm from TS (left).
			Temperatures are still being overestimated by a factor of $\sim$1.6, which is an improvement over the previous factor of 4. 
			}
		\label{fig:mag-axial-scan}
		\vspace{-10pt}
	\end{figure}
	The $T_e$ and $n_e$ measurements on the S probe show a rise in temperature and a drop in density as the probes are moved along the beam closer to the source. 
	It was first thought that, due to the S probe being the most central and therefore closest to the peak $T_e$ and $n_e$, this drop in density closer to the source was the beam profile narrowing closer to the source. 
	Upon re-examination of the properties of the beam however, the plasma should be well confined along the whole range of linear positions measured at, so the explanation is more likely to be that the TEAC arm, which is holding the probes and pushing them into position, does not perfectly align down the centre of the plasma beam. 
	Therefore this axial profile measurement gives a good measure of the radial movement of the probes with axial distance. 
	The temperature could still be expected to drop along the beam axis due to radiative processes removing energy from the system. 
	The absolute values of the $T_e$ and $n_e$ measurements is over and underestimating compared to the TS measurement for temperature and density respectively. 
	The temperature is being overestimated by a factor of $\sim$1.6, which an improvement over the previous factor of $\sim$4 but still leaves improvements to be made either in greater understanding of the physical picture or reassessment of the calibration. 
	It should also be noted that the temperature, as measured by the TS system, is still under 5eV and so not ideal for measurement with LPs.
	The density underestimation, by a factor of up to $\sim$6, has no clear explanation. 
	It could be due: to an error in the calculation of the probe's collection area; an incorrect assumption of ion-electron temperature equivalence in the calculation of the sound speed, although Collective TS measurements taken recently on Magnum-PSI indicate this is unlikely; or, as for the temperature, some unaccounted for physics or incorrect calibration.
	The voltage and current calibration will be reevaluated following these findings.
	The jump in density seen at linear position 25mm is likely a sudden change in plasma parameters caused by some physical change within the source - it is unfortunately impossible to verify this however due to the necessary absence of TS coverage throughout the scan. 
	
	Further analysis of different shots is currently underway. 
	There has been a twofold approach in separating the shots, with half being magnetics scans (i.e. magnetic field incidence scans, magnetic field strength scans) and half looking at species effects (i.e. different species plasmas, neutral puffing etc.).
	
%------------------------------------------------------------------------------------
%	Langmuir Probes
%------------------------------------------------------------------------------------

\section{\label{sec:lpsims}Simulations of Langmuir Probes}

\subsection{\label{ssec:sim_setup}General Simulation Setup}
	All simulations referred to in the following section were 2-dimensional particle-in-cell simulations of the plasma around a probe tip.
	The simulations all had periodic boundary conditions on the left and right and a particle source across the top of the domain. 
	The particle source is coded in such a way as to re-inject particles leaving the domain to not disturb the temperature of the plasma.
	The plasma was allowed to come to a steady state - designated as 2 simulation cycles, which in SPICE are defined by the window traversal time for an ion - and then a voltage sweep is performed from -220V to 220V to pre-designated probe objects in the simulation window. 
	Current is measured as the voltage is swept, allowing an IV characteristic to be extracted.
	This IV characteristic can then be fit with \cref{eq:ivcurvesheathexp}, performed using the non-linear least squares fitting function curve\_fit() (part of the optimise package in numpy).
	The parameters extracted by this fit ($T_e, n_e, a, I_{i, sat}$) represent the plasma parameters measured by a perfectly conducting probe with no capacitance effects. 
	Note that this analysis has been generalised \cite{LelandGithub} and was also applied to the data from experimental probes in \cref{sec:mag}.

\subsection{\label{ssec:marconi}Move to Marconi and 24-hour Time Limits}
    Following the benchmarking period on CUMULUS in 2018, time was applied for in the 2019-2020 allocation on Marconi. 
    This application was successful, with $\sim10^{7}$ core-hours allocated, so work was started in January of 2019 to get SPICE-2 simulations working on Marconi.
    This was unfortunately dogged by months of hardware problems on Marconi and then issues with running SPICE-2 in the 24 hour time limit imposed by Marconi.
    
    This 24 hour time limit is in place to encourage high parallelisability of code, but SPICE-2 is unfortunately limited by its solver which, while recently parallelised, offers only limited scalability - peaking at around 100 cores.
    % The code is also very memory intensive, which is the main bottleneck.
    This means that most SPICE-2 simulations, dependent principally on density and the incidence angle of the magnetic field, will take over 24 hours, and often many times more than this. 
    Therefore the simulations must be continually restarted in order finish when using on Marconi. 
    Unfortunately the current version of SPICE-2 is unable to effectively restart while maintaining a measure of the current to a simulated probe. 
    This is a recognised bug, but as the support team for SPICE-2 is small, and I am the only user who needs this feature to work, the resolution has yet to be found or implemented.
    This is however ongoing and a workaround has been found which allows new simulations to be joined together and now allows longer simulations involving probes to be effectively analysed. 
    
    % A new python package has also been created to assist in the submission of multiple 24 hour SPICE-2 simulations.
    % This package can automatically submit jobs on a cluster with batch job submission and have them automatically restart until a certain time limit is met. 
    % This package can also submit parameter scans, i.e. it can read an input file with $n$ values for a particular parameter and submit $m$ jobs for each parameter, where m is the required number of restarts.
    % There is also automatic logging functionality to keep track of the simulations submitted along with their parameters and machine/job information. 
    % This package is currently only functional for SPICE-2 and SPICE-3, but has been written to be code agnostic and work is underway to make it a general, open-source tool. 
    % More details can be found on the \hyperlink{https://github.com/jackleland/autospice/tree/development}{github}.
	
\subsection{\label{subsec:mastu_probe_design} Progression of MAST-U Probe Design}
	Due to the limitations on running simulations of probes, only lower density simulations could be performed as density is the main scaling factor for the length of a SPICE simulation.
	Lowering density to $1 \times 10^{18} \; m^{-3}$ for a simulation with an incidence angle of $10^{\circ}$ resulted in a completion time of $\sim$10 hours.
	The completion time would be larger for lower angles of magnetic field incidence angle, as the length of the simulation scales with $\frac{1}{\sin{\theta}}$. 
	
	A study of the individual aspects of the probe design from MAST-U was carried out in 2 dimensions, using lower density simulations. 
	The difference between the previously run benchmarking simulations of Murphy-Sugrue, which consisted of a flush probe with no shadowing of the leading edge, and the desired simulations of the MAST-U standard probe tip, were summarised with the following features: (a) an angled tip, angled 10$^{\circ}$ to the tile, (b) a recessed probe body which allowed for shadowing of the leading edge of the probe, and (c) a recessed rear tile, set 1mm down from the fore-tile.  
	Thus 4 different probe tip geometries were simulated at a range of angles between 1$^{\circ}$ and 90$^{\circ}$, to see the effect that each of the features had on the measurement of the temperature and density at different angles of magnetic field incidence ($\theta_{\perp}$).
	The plasma parameters used were: $T_e$ = 5eV, $n_e$ = $1 \times 10^{18} \; m^{-3}$, $B$ = 0.5T and $\theta_{\perp}$ as specified above. 
	A fifth tip geometry, that of the S probe from the DSF, was also simulated. 
	The results of this angle scan and the effect of the $T_e$ and $n_e$ measurements for each of the simulated probes can be seen in figure \ref{fig:sim-incidence-angle}.
	\begin{figure}[!tb]
		\center{
% 			\vspace{-20pt}
			\includegraphics[width=0.9\linewidth]{simulation-incidence_angle.png}
		}
		\vspace{-0pt}
		\caption{
			Graph of the simulation geometry for the flush and angled probe tips (Left) and the measured normalised temperature for a simulated probe with corresponding flush and angled tip geometry. 
			In the flush case the measured temperatures are less accurate at shallower incidence angles, which is remedied somewhat by the angled probe.
			}
		\label{fig:sim-incidence-angle}
		\vspace{-10pt}
	\end{figure}
	
	The simulations appear to show that for the flush case, there is an increased overestimation of $T_e$ as the $\theta_{\perp}$ approaches 0$^{\circ}$, which is remedied somewhat by angling the probe tip 10$^{\circ}$ relative to the tile. 
	This effect appears to work at grazing angles of incidence, but not at the very oblique angles of incidence.
	There also appears to be a systematic overestimation of temperature around the 15$^{\circ}$ point using an angled probe compared to a flush probe. 
	The results on the effect of recessing the probe relative to the fore-tile and the recessing of the rear-tile are in the final stages of being analysed, but these are not expected to deviate greatly from the simple angled case.
	
    These simulations will form the basis for simulations of the MAST-U and DSF probes going forward into higher densities and dimensionality. 

%------------------------------------------------------------------------------------
%	PhD Project Plan
%------------------------------------------------------------------------------------

\section{\label{sec:phd}Future Work \& Thesis Plan}
    The major experiment of the PhD has been completed this year and the results from this experiment will form the basis of two chapters in the thesis. 
    Unfortunately, due to the MAST-U timeline being pushed back and the start date of the physics campaign moving to May 2020, it is unclear whether MAST-U can be relied upon for providing data for comparison with MAST-U relevant simulations. 
    The computational work in the thesis will therefore consist of simulations for MAST-U probes and the DSF probes and focus on the improvement of the FMP model. 
    
	The overall plan for the remaining 9 months of research, i.e. Dec 2019 - Sep 2020, has been drafted in \cref{fig:timeline}.
	\begin{figure}[t]
		\center{
			\vspace{-15pt}
			\includegraphics[width=0.6\linewidth]{phd_plan_2020.png}
		}
		\vspace{-0pt}
		\caption{A Gantt chart denoting breakdown of proposed work and milestones to achieve. The Magnum-PSI analysis needs to be finalised before attending the PSI conference in Jeju 2020. Squares marked A denote the submission deadline for the respective conference abstract.}
		\label{fig:timeline} 
	\end{figure}
	This is not finalised but is intended to give an overall impression of the desired work and how long it will take. 
	It primarily focuses on finalising the analysis of Magnum-PSI probe data and comparing this to simulations of the DSF probe head. 
	Focus will also be maintained on the simulation of MAST-U Langmuir probes in MAST-U relevant plasmas and the extension of the Langmuir probe model to 3 dimensions. 
	
	Two conferences are currently planned to be attended: plasma surface interactions (PSI) and the IoP plasma physics conference in 2020.
	I plan to present a poster at PSI, and as part of the conference I will utilise the opportunity to submit a paper for publication in the conference proceedings journal.
	Other conferences will be considered, such as the EPS conference in Europe, depending on available budget and time.
	It is intended for papers to be published with the data taken at Magnum-PSI, but completing the work for the thesis will take priority. 



%----------------------------------------------------------------------------------------
%	BIBLIOGRAPHY
%----------------------------------------------------------------------------------------

\begingroup
\setstretch{0.8}
\setlength\bibitemsep{3.5pt}
\printbibliography
\endgroup
%
%\bibliography{sample}

%----------------------------------------------------------------------------------------

\end{document}